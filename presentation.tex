\documentclass[11pt]{beamer}
\usepackage[utf8]{inputenc}
\usepackage[T2A]{fontenc}
\usepackage[english,russian]{babel}
\usepackage{amsmath}
\usepackage{amsfonts}
\usepackage{amssymb}
\usepackage{graphicx}
\usepackage{listings}
\usepackage{xcolor}
\usepackage{tikz}
\usepackage{ragged2e}
\usetikzlibrary{shapes,arrows,positioning}

% Цветовая схема
\usetheme{Madrid}
\usecolortheme{seahorse}

% Уменьшаем отступы и размеры
\setbeamersize{text margin left=3mm, text margin right=3mm}
\setlength{\leftmargini}{1.5em}
\setlength{\leftmarginii}{1em}
\setbeamertemplate{itemize item}{\tiny\raise0.5pt\hbox{$\bullet$}}
\setbeamertemplate{itemize subitem}{\tiny\raise0.5pt\hbox{$\circ$}}

\title{Теория Демпстера-Шейфера в управлении ИТ-проектами}
\author{Плеханов Е.С.}
\institute{СПбПУ, ИКНТ ВШ ПИ}
\date{2025}

\begin{document}

\begin{frame}[plain]
    \centering
    \vspace{0.5cm}
    
    {\footnotesize Минобрнауки России \\
    Санкт-Петербургский политехнический университет Петра Великого \\
    Институт компьютерных наук и кибербезопасности \\
    Высшая школа программной инженерии}
    
    \vspace{0.5cm}
    
    {\Large \textbf{Презентация на тему:}} \\
    \vspace{0.3cm}
    {\LARGE \textbf{Теория Демпстера-Шейфера}} \\
    {\LARGE \textbf{в управлении ИТ-проектами}}
    
    \vspace{0.5cm}
    
    {\large по дисциплине "Методы принятия решений"}
    
    \vspace{0.5cm}
    
    \begin{minipage}[t]{0.45\textwidth}
        \raggedright
        \footnotesize
        Руководитель: \\
        старший преподаватель \\
        В.А. Пархоменко
    \end{minipage}
    \hfill
    \begin{minipage}[t]{0.45\textwidth}
        \raggedleft
        \footnotesize
        Выполнил: \\
        студент группы 5140903/40101 \\
        Плеханов Е.С.
    \end{minipage}
    
    \vspace{0.5cm}
    
    {\large Санкт-Петербург \\
    2025}
\end{frame}

% Слайд 2: Структура доклада
\begin{frame}{Структура доклада}
    \footnotesize
    \begin{enumerate}
        \item Проблема неопределенности в ИТ-проектах
        \item Основные понятия теории Демпстера-Шейфера
        \item Пример 2.1: Кандидаты на должность
        \item Пример 2.6: Комбинирование экспертных оценок
        \item Правило Ягера как альтернатива
        \item Практическая реализация
        \item Выводы и применение в ИТ
    \end{enumerate}
\end{frame}

% Слайд 3: Проблема неопределенности
\begin{frame}{Проблема неопределенности в ИТ-проектах}
    \footnotesize
    \begin{block}{Традиционный подход}
        \begin{itemize}
            \item Точечные оценки: ``проект займет 3 месяца''
            \item Игнорирование неопределенности
            \item Субъективное усреднение мнений
        \end{itemize}
    \end{block}
    
    \begin{block}{Реальность ИТ-проектов}
        \begin{itemize}
            \item Эксперты дают разные оценки
            \item Противоречивые мнения о рисках
            \item Необходимость объективного комбинирования
        \end{itemize}
    \end{block}
    
    \begin{alertblock}{Решение}
        \textbf{Теория Демпстера-Шейфера}:
        \begin{itemize}
            \item Интервальные вероятности
            \item Учет незнания
            \item Математическое комбинирование
        \end{itemize}
    \end{alertblock}
\end{frame}

% Слайд 4: Основные понятия
\begin{frame}{Основные понятия теории Демпстера-Шейфера}
    \scriptsize
    \begin{block}{Фрейм discernment}
        \[
        \Omega = \{\omega_1, \omega_2, \dots, \omega_n\}
        \]
        Множество всех возможных исходов
    \end{block}
    
    \begin{block}{Базовая вероятностная функция (BPA)}
        \[
        m: 2^{\Omega} \rightarrow [0,1], \quad m(\emptyset) = 0, \quad \sum m(A) = 1
        \]
    \end{block}
    
    \begin{block}{Функция доверия (Belief)}
        \[
        \text{Bel}(A) = \sum_{B \subseteq A} m(B)
        \]
    \end{block}
    
    \begin{block}{Функция правдоподобия (Plausibility)}
        \[
        \text{Pl}(A) = \sum_{B \cap A \neq \emptyset} m(B)
        \]
    \end{block}
\end{frame}

% Слайд 5: Пример 2.1 - Условие
\begin{frame}{Пример 2.1: Кандидаты на должность}
    \footnotesize
    \begin{block}{Условие задачи}
        \textbf{4 кандидата}: $\Omega = \{1, 2, 3, 4\}$ \\
        \textbf{10 экспертов}:
        
        \begin{tabular}{|l|c|c|}
            \hline
            Мнение экспертов & Множество & Количество \\
            \hline
            ``Кандидат 1'' & $\{1\}$ & 5 \\
            ``Кандидат 1 или 2'' & $\{1,2\}$ & 2 \\
            ``Кандидат 3'' & $\{3\}$ & 3 \\
            \hline
        \end{tabular}
    \end{block}
    
    \begin{block}{Базовые вероятности}
        \[
        m(\{1\}) = 0.5, \quad m(\{1,2\}) = 0.2, \quad m(\{3\}) = 0.3
        \]
    \end{block}
\end{frame}

% Слайд 6: Пример 2.1 - Вычисления
\begin{frame}{Пример 2.1: Вычисления функций доверия}
    \footnotesize
    \begin{block}{Результаты для каждого кандидата}
        \begin{itemize}
            \item \textbf{Кандидат 1:}
            $\text{Bel}=0.5$, $\text{Pl}=0.7$, Интервал: $[0.5, 0.7]$
            
            \item \textbf{Кандидат 2:}
            $\text{Bel}=0.0$, $\text{Pl}=0.2$, Интервал: $[0.0, 0.2]$
            
            \item \textbf{Кандидат 3:}
            $\text{Bel}=0.3$, $\text{Pl}=0.3$, Интервал: $[0.3, 0.3]$
            
            \item \textbf{Кандидат 4:}
            $\text{Bel}=0.0$, $\text{Pl}=0.0$, Интервал: $[0.0, 0.0]$
        \end{itemize}
    \end{block}
    
    \begin{block}{Интерпретация}
        \begin{itemize}
            \item Кандидат 1: уверенность + неопределенность
            \item Кандидат 2: ``может быть'' (не знаем)
            \item Кандидат 3: полная определенность
            \item Кандидат 4: невозможно
        \end{itemize}
    \end{block}
\end{frame}


% Слайд 7: Комбинирование свидетельств
\begin{frame}{Комбинирование свидетельств: Проблема ИТ-проекта}
    \footnotesize
    \begin{block}{Типичная ситуация в ИТ}
        \begin{itemize}
            \item \textbf{Команда разработки}: ``3-4 месяца''
            \item \textbf{Команда тестирования}: ``4-5 месяцев''
            \item \textbf{Менеджер}: ``2 месяца''
        \end{itemize}
    \end{block}
    
    \begin{alertblock}{Проблема}
        Как объективно объединить противоречивые оценки?
    \end{alertblock}
    
    \begin{block}{Решение}
        \textbf{Правило комбинирования Демпстера}:
        \begin{itemize}
            \item Математическое объединение
            \item Учет конфликтов
            \item Объективный результат
        \end{itemize}
    \end{block}
\end{frame}

% Слайд 8: Пример 2.6 - Условие
\begin{frame}{Пример 2.6: Комбинирование экспертных оценок}
    \scriptsize
    \begin{block}{Условие задачи}
        \textbf{Предприятия}: $\Omega = \{1, 2, 3, 4\}$
        
        \textbf{Источник 1} (8 экспертов):
        \begin{itemize}
            \item 5 экспертов: предприятие 1 $\rightarrow \{1\}$
            \item 3 эксперта: предприятие 2 или 3 $\rightarrow \{2,3\}$
        \end{itemize}
        
        \textbf{Источник 2} (16 экспертов):
        \begin{itemize}
            \item 8 экспертов: предприятие 1 или 2 $\rightarrow \{1,2\}$
            \item 7 экспертов: предприятие 3 $\rightarrow \{3\}$
            \item 1 эксперт: предприятие 4 $\rightarrow \{4\}$
        \end{itemize}
    \end{block}
    
    \begin{block}{Базовые вероятности}
        \[
        m_1(\{1\}) = 0.625, \quad m_1(\{2,3\}) = 0.375
        \]
        \[
        m_2(\{1,2\}) = 0.5, \quad m_2(\{3\}) = 0.4375, \quad m_2(\{4\}) = 0.0625
        \]
    \end{block}
\end{frame}

% Слайд 9: Правило Демпстера
\begin{frame}{Правило комбинирования Демпстера}
    \scriptsize
    \begin{block}{Математическая формула}
        \[
        m_{12}(A) = \frac{1}{1-K} \sum_{B \cap C = A} m_1(B) \cdot m_2(C)
        \]
        где 
        \[
        K = \sum_{B \cap C = \emptyset} m_1(B) \cdot m_2(C)
        \]
    \end{block}
    
    \begin{block}{Псевдокод алгоритма}
        \begin{enumerate}
            \item Вычислить пересечения $B \cap C$
            \item Найти конфликт $K$
            \item Для каждого $A$: вычислить $m_{12}(A)$
            \item Нормализовать на $(1-K)$
        \end{enumerate}
    \end{block}
\end{frame}

% Слайд 10: Пример 2.6 - Вычисления
\begin{frame}{Пример 2.6: Вычисления}
    \tiny
    \begin{block}{Таблица пересечений}
        \centering
        \begin{tabular}{|c|c|c|}
            \hline
            & $m_1(\{1\}) = 0.625$ & $m_1(\{2,3\}) = 0.375$ \\
            \hline
            $m_2(\{1,2\}) = 0.5$ & $\{1\}$ & $\{2\}$ \\
            \hline
            $m_2(\{3\}) = 0.4375$ & $\emptyset$ & $\{3\}$ \\
            \hline
            $m_2(\{4\}) = 0.0625$ & $\emptyset$ & $\emptyset$ \\
            \hline
        \end{tabular}
    \end{block}
    
    \begin{block}{Вычисление конфликта}
        \[
        K = 0.625 \cdot 0.4375 + 0.625 \cdot 0.0625 + 0.375 \cdot 0.0625 = 0.336
        \]
    \end{block}
    
    \begin{block}{Результат комбинирования}
        \[
        m_{12}(\{1\}) = 0.4706, \quad m_{12}(\{2\}) = 0.2824, \quad m_{12}(\{3\}) = 0.2470
        \]
    \end{block}
\end{frame}

% Слайд 11: Парадокс Заде
\begin{frame}{Проблема правила Демпстера: Парадокс Заде}
    \footnotesize
    \begin{block}{Медицинский пример}
        \textbf{Врач 1:}
        \begin{itemize}
            \item Менингит: $m=0.99$
            \item Опухоль мозга: $m=0.01$
        \end{itemize}
        
        \textbf{Врач 2:}
        \begin{itemize}
            \item Сотрясение: $m=0.99$
            \item Опухоль мозга: $m=0.01$
        \end{itemize}
    \end{block}
    
    \begin{alertblock}{Результат комбинирования}
        \[
        \text{Bel}(опухоль) = 1.0
        \]
        Оба врача считали опухоль маловероятной!
    \end{alertblock}
    
    \begin{block}{Проблема}
        Правило Демпстера усиливает редкие совпадения
    \end{block}
\end{frame}

% Слайд 12: Правило Ягера
\begin{frame}{Правило Ягера: Альтернативный подход}
    \scriptsize
    \begin{block}{Философия Ягера}
        ``Конфликт → незнание, а не усиление''
    \end{block}
    
    \begin{block}{Математическая формула}
        \[
        q(A) = \sum_{B \cap C = A} m_1(B) \cdot m_2(C)
        \]
        \[
        m_{Yag}(A) = q(A) \quad \text{для } A \neq \Omega, A \neq \emptyset
        \]
        \[
        m_{Yag}(\Omega) = q(\Omega) + q(\emptyset)
        \]
    \end{block}
    
    \begin{block}{Ключевое отличие}
        Конфликт переносится в универсальное множество $\Omega$ (``не знаю'')
    \end{block}
\end{frame}

% Слайд 13: Сравнение методов
\begin{frame}{Сравнение Демпстера и Ягера}
    \scriptsize
    \begin{block}{Результаты для примера 2.6}
        \centering
        \begin{tabular}{|c|c|c|c|}
            \hline
            Множество & Демпстер & Ягер & Разница \\
            \hline
            $\{1\}$ & 0.4706 & 0.3125 & -0.1581 \\
            $\{2\}$ & 0.2824 & 0.1875 & -0.0949 \\
            $\{3\}$ & 0.2470 & 0.1641 & -0.0829 \\
            $\Omega$ & 0.0000 & 0.3360 & +0.3360 \\
            \hline
        \end{tabular}
    \end{block}
    
    \begin{block}{Интерпретация}
        \begin{itemize}
            \item \textbf{Демпстер}: усиливает согласованные свидетельства
            \item \textbf{Ягер}: признает незнание (33.6\% → ``не знаю'')
        \end{itemize}
    \end{block}
    
    \begin{block}{Когда что использовать?}
        \begin{itemize}
            \item Демпстер: надежные источники, нужно решение
            \item Ягер: ненадежные источники, важна осторожность
        \end{itemize}
    \end{block}
\end{frame}

% Слайд 14: Практическая реализация
\begin{frame}{Практическая реализация}
    \scriptsize
    \begin{block}{Архитектура программы}
        \begin{itemize}
            \item \texttt{main.py} - главный файл приложения
            \item \texttt{dempster\_core.py} - ядро теории Д-Ш
            \item \texttt{examples.py} - реализация примеров из книги
            \item \texttt{visualizer.py} - визуализация результатов
        \end{itemize}
    \end{block}

    \begin{minipage}{0.9\textwidth}
    \begin{block}{Основные возможности}
        \begin{itemize}
            \item Консольное меню для выбора примеров
            \item Автоматическая визуализация результатов
            \item Сравнение методов комбинирования
            \item Подробные текстовые отчеты
        \end{itemize}
    \end{block}
    
    \begin{exampleblock}{Пример вывода программы}
        \scriptsize
        \texttt{>>> Пример 2.1: Кандидаты на должность} \\
        \texttt{Базовые вероятности: m(\{1\})=0.5, m(\{1,2\})=0.2, m(\{3\})=0.3} \\
        \texttt{Кандидат 1: Bel=0.5, Pl=0.7, Интервал: [0.5, 0.7]} \\
        \texttt{Кандидат 2: Bel=0.0, Pl=0.2, Интервал: [0.0, 0.2]}
    \end{exampleblock}

    \end{minipage}
\end{frame}

% Слайд 15: Применение в ИТ
\begin{frame}{Применение в управлении ИТ-проектами}
    \footnotesize
    \begin{block}{Оценка сроков проекта}
        \begin{itemize}
            \item \textbf{Bel(успех)} - пессимистичная оценка
            \item \textbf{Pl(успех)} - оптимистичная оценка  
            \item \textbf{Pl - Bel} - уровень неопределенности
        \end{itemize}
    \end{block}
    
    \begin{block}{Анализ рисков}
        \begin{itemize}
            \item Комбинирование оценок разных экспертов
            \item Объективная обработка конфликтов
            \item Количественная оценка неопределенности
        \end{itemize}
    \end{block}
    
    \begin{block}{Принятие решений}
        \begin{itemize}
            \item Стратегии для разных уровней уверенности
            \item Учет ``не знаю'' как valid состояния
            \item Более обоснованные управленческие решения
        \end{itemize}
    \end{block}
\end{frame}

% Слайд 16: Выводы
\begin{frame}{Выводы}
    \footnotesize
    \begin{block}{Преимущества теории Демпстера-Шейфера}
        \begin{itemize}
            \item Работа с неполной информацией
            \item Интервальные оценки вместо точечных
            \item Объективное комбинирование мнений
            \item Учет конфликтов между источниками
        \end{itemize}
    \end{block}
    
    \begin{block}{Ограничения}
        \begin{itemize}
            \item Сложность вычислений для больших фреймов
            \item Выбор правила комбинирования требует обоснования
            \item Интерпретация результатов требует опыта
        \end{itemize}
    \end{block}
    
    \begin{block}{Перспективы в ИТ}
        \begin{itemize}
            \item Системы поддержки принятия решений
            \item Управление рисками проектов
            \item Анализ надежности complex systems
        \end{itemize}
    \end{block}
\end{frame}

% Слайд 17: Источники
\begin{frame}{Источники}
    \footnotesize
    \begin{block}{Литература}
        \begin{itemize}
            \item \textbf{Уткин Л.В.} ``Анализ риска и принятие решений при неполной информации'' - Глава 2
        \end{itemize}
    \end{block}
    
    \begin{block}{Программная реализация}
        \begin{itemize}
            \item \textbf{GitHub репозиторий}: https://github.com/EgorPlehanov/dempster-shafer-theory
            \item Полный исходный код всех примеров
            \item Документация и инструкции по установке
            \item Тестовые данные и примеры использования
        \end{itemize}
    \end{block}
    
    \begin{block}{Технологии}
        \begin{itemize}
            \item Python 3.8+, Matplotlib, NumPy, Pandas, Seaborn
            \item LaTeX для документации и презентации
            \item Git для контроля версий
        \end{itemize}
    \end{block}
\end{frame}

% Слайд 18: Вопросы
\begin{frame}{Вопросы и ответы}
    \centering
    \vspace{1cm}
    \Huge \textbf{Спасибо за внимание!}
    
    \vspace{1cm}
    \Large Вопросы?

\end{frame}

\end{document}