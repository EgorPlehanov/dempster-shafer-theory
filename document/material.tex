\documentclass[a4paper,10pt]{article}
\usepackage[utf8]{inputenc}
\usepackage[T2A]{fontenc}
\usepackage[russian]{babel}
\usepackage{amsmath}
\usepackage{amssymb}
\usepackage{geometry}
\usepackage{algorithm}
\usepackage{algpseudocode}

\geometry{left=2cm,right=2cm,top=2cm,bottom=2cm}

\begin{document}

\begin{center}
\textbf{РАЗДАТОЧНЫЙ МАТЕРИАЛ} \\
\textbf{Теория Демпстера-Шейфера} \\
Основные термины и формулы \\
Студент: Плеханов Е.С., группа 5140903/40101
\end{center}

\section*{Основные понятия}

\begin{itemize}
\item \textbf{Фрейм discernment}: $\Omega = \{\omega_1, \omega_2, \dots, \omega_n\}$
\item \textbf{Базовая вероятность (BPA)}: $m: 2^{\Omega} \rightarrow [0,1]$
\item \textbf{Функция доверия}: $\text{Bel}(A) = \sum_{B \subseteq A} m(B)$
\item \textbf{Функция правдоподобия}: $\text{Pl}(A) = \sum_{B \cap A \neq \emptyset} m(B)$
\item \textbf{Интервал вероятности}: $[\text{Bel}(A), \text{Pl}(A)]$
\end{itemize}

\section*{Правило комбинирования Демпстера}

\begin{algorithm}
\caption{Правило Демпстера}
\begin{algorithmic}[1]
\Require Две BPA: $m_1$ и $m_2$
\Ensure Комбинированная BPA: $m_{12}$
\State $K \gets 0$
\For{всех $B$ в фокальных элементах $m_1$}
    \For{всех $C$ в фокальных элементах $m_2$}
        \If {$B \cap C = \emptyset$}
            \State $K \gets K + m_1(B) \cdot m_2(C)$
        \EndIf
    \EndFor
\EndFor
\State $Z \gets 1 - K$
\For{всех непустых $A \neq \emptyset$}
    \State $m_{12}(A) \gets 0$
    \For{всех пар $(B,C)$: $B \cap C = A$}
        \State $m_{12}(A) \gets m_{12}(A) + m_1(B) \cdot m_2(C)$
    \EndFor
    \State $m_{12}(A) \gets m_{12}(A) / Z$
\EndFor
\State \Return $m_{12}$
\end{algorithmic}
\end{algorithm}

\section*{Правило Ягера}

\begin{algorithm}
\caption{Правило Ягера}
\begin{algorithmic}[1]
\Require Две BPA: $m_1$ и $m_2$
\Ensure Комбинированная BPA: $m_{Yag}$
\For{всех $B$ в фокальных элементах $m_1$}
    \For{всех $C$ в фокальных элементах $m_2$}
        \State $A \gets B \cap C$
        \If {$A \neq \emptyset$}
            \State $m_{Yag}(A) \gets m_{Yag}(A) + m_1(B) \cdot m_2(C)$
        \Else
            \State $m_{Yag}(\Omega) \gets m_{Yag}(\Omega) + m_1(B) \cdot m_2(C)$
        \EndIf
    \EndFor
\EndFor
\State \Return $m_{Yag}$
\end{algorithmic}
\end{algorithm}

\section*{Пример 2.1: Кандидаты на должность}

\textbf{Данные:} $\Omega = \{1,2,3,4\}$, $N=10$ экспертов

\begin{tabular}{|l|c|c|}
\hline
Мнение экспертов & Множество & m(A) \\
\hline
``Кандидат 1'' & $\{1\}$ & 0.5 \\
``Кандидат 1 или 2'' & $\{1,2\}$ & 0.2 \\
``Кандидат 3'' & $\{3\}$ & 0.3 \\
\hline
\end{tabular}

\textbf{Результаты:}
\begin{itemize}
\item Кандидат 1: $\text{Bel}=0.5$, $\text{Pl}=0.7$, $[0.5, 0.7]$
\item Кандидат 2: $\text{Bel}=0.0$, $\text{Pl}=0.2$, $[0.0, 0.2]$
\item Кандидат 3: $\text{Bel}=0.3$, $\text{Pl}=0.3$, $[0.3, 0.3]$
\item Кандидат 4: $\text{Bel}=0.0$, $\text{Pl}=0.0$, $[0.0, 0.0]$
\end{itemize}

\section*{Пример 2.6: Комбинирование свидетельств}

\textbf{Источник 1:} $m_1(\{1\})=0.625$, $m_1(\{2,3\})=0.375$

\textbf{Источник 2:} $m_2(\{1,2\})=0.5$, $m_2(\{3\})=0.4375$, $m_2(\{4\})=0.0625$

\textbf{Конфликт:} $K = 0.336$

\textbf{Результат Демпстера:}
\begin{itemize}
\item $m_{12}(\{1\}) = 0.4706$
\item $m_{12}(\{2\}) = 0.2824$
\item $m_{12}(\{3\}) = 0.2470$
\end{itemize}

\textbf{Результат Ягера:}
\begin{itemize}
\item $m_{Yag}(\{1\}) = 0.3125$
\item $m_{Yag}(\{2\}) = 0.1875$
\item $m_{Yag}(\{3\}) = 0.1641$
\item $m_{Yag}(\Omega) = 0.3360$
\end{itemize}

\section*{Ключевые отличия}

\begin{itemize}
\item \textbf{Демпстер}: конфликт → нормализация (усиление согласия)
\item \textbf{Ягер}: конфликт → незнание (осторожность)
\item $\text{Pl}(A) - \text{Bel}(A)$ = мера неопределенности
\item $\text{Bel}(A) = 1 - \text{Pl}(A^c)$
\end{itemize}

\end{document}